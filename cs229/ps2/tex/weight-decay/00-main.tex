\item \points{15} {\bf Bayesian Logistic Regression and weight decay}

% For problem 3
\newcommand{\thetaml}{{\theta_{\rm ML}}}
\newcommand{\thetamap}{{\theta_{\rm MAP}}}

  Consider using a logistic regression model $h_\theta(x) =
  g(\theta^Tx)$ where $g$ is the sigmoid function, and let a training
  set $\{(\xsi, \ysi); i=1, \ldots, \nexp\}$ be given as usual.  The maximum
  likelihood estimate of the parameters $\theta$ is given by
  \[
  \thetaml = \arg \max_\theta \prod_{i=1}^\nexp p(\ysi | \xsi ; \theta).
  \]
  If we wanted to regularize logistic regression, then we might put a
  Bayesian prior on the parameters.  Suppose we chose the prior $\theta
  \sim \calN(0,\tau^2I)$ (here, $\tau > 0$, and $I$ is the
  $\di+1$-by-$\di+1$ identity matrix), and then found the MAP estimate of
  $\theta$ as:
  \[
  \thetamap = \arg \max_\theta p(\theta) \prod_{i=1}^\nexp p(\ysi | \xsi ,
  \theta)
  \]

  Prove that
  \[
  ||\thetamap||_2 \leq ||\thetaml||_2
  \]
  [Hint: Consider using a proof by contradiction.]\\
  {\bf Remark.}  For this reason, this form of regularization is
  sometimes also called {\bf weight decay}, since it encourages the
  weights (meaning parameters) to take on generally smaller values.

\ifnum\solutions=1 {
  
\begin{answer}
\end{answer}

} \fi
